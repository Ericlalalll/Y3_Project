{
\setstretch{1.2}
\begin{abstract}
\textcolor{black}{The community detection problem has gained significant attention across various scientific communities due to its substantial impact on numerous real-world applications. This project studies the community detection problem under the stochastic block model, particularly focusing on the hard regime of the Kesten-Stigum bound conjecture. Given the inherent hardness of this regime, we introduce a special setting called the partial seed set, which consists of a set of vertices whose community membership is known beforehand. We propose a greedy recovery algorithm that solves the reconstruction task in the hard regime under the partial seed set setting. The effectiveness of the algorithm is evaluated through extensive experiments and rigorous mathematical analysis. The proposed method also holds considerable practical relevance, especially in real-world scenarios where certain vertex memberships are known a priori, such as the political leaders in a political ideologies network. Our findings contribute to the development of efficient and accurate community detection methods that leverage available information to uncover the underlying community structure in complex networks, especially when the community structure is weak.}
\vspace{3mm}\\
\textbf{\textcolor{Fuchsia}{Keywords:}} \textit{community detection, graph clustering, graph partition, algorithms, graph theory, network science}
\end{abstract}
}