
\usepackage[]{geometry}
\usepackage[utf8]{inputenc}
\usepackage[UKenglish]{babel}
\usepackage[UKenglish]{isodate}
\usepackage{amsmath}
\usepackage{amsfonts}
\usepackage{amssymb}
\usepackage{amsthm}
\usepackage{csquotes}
\usepackage{mathrsfs}
\usepackage{float}
\usepackage{graphicx}
\usepackage{chngpage}
\usepackage{blkarray}
\usepackage{calc}
\PassOptionsToPackage{hyphens}{url}
\usepackage{fancyhdr}
\usepackage{hyperref}
\usepackage{setspace}
\usepackage[dvipsnames]{xcolor}
\usepackage{titletoc}
\usepackage[explicit]{titlesec}
\usepackage[linesnumbered, ruled,vlined]{algorithm2e}
\usepackage{algpseudocode}
\newcommand{\Input}[1]{\State \textbf{Input:} #1}
\newcommand{\Output}[1]{\State \textbf{Output:} #1}
\usepackage{appendix} %Take a wild guess slick
\usepackage{caption}
\captionsetup{
  font={small,stretch=1.3},
  skip=5pt,
  textfont={color=darkgray}
}

\usepackage{tikz}
\usetikzlibrary{arrows.meta, positioning, shapes.misc, calc, tikzmark}


\titleformat{\chapter}[display]
  {\normalfont\Huge\bfseries\color{Fuchsia}} % Common formatting for the title
  {\flushright\Large\sffamily\color{Fuchsia}\MakeUppercase{\chaptertitlename}\hspace{1ex}{\fontsize{60}{60}\selectfont\thechapter}} % Formatting for 'Chapter 1'
  {20pt} % Space between the number and title text
  {\Huge #1} % Here the actual title text will be inserted




\titleformat{\section}
  {\normalfont\Large\bfseries\color{Fuchsia}}
  {\thesection}{1em}{#1}
  \titleformat{\subsection}
  {\normalfont\large\bfseries\color{Fuchsia}}
  {\thesubsection}{1em}{#1}

\usepackage[backend=biber, style=alphabetic]{biblatex}
\addbibresource{References.bib}
\usepackage[sc]{mathpazo}

\usepackage[T1]{fontenc}

\usepackage{listings}
\definecolor{darkred}{rgb}{0.545, 0, 0}
\definecolor{codegreen}{rgb}{0.25,0.5,0.35}  % 深绿色
\definecolor{codegray}{rgb}{0.6,0.6,0.6}     % 深灰色
\definecolor{backcolour}{rgb}{0.95,0.95,0.95} % 浅灰色背景

\lstdefinestyle{mystyle}{
    backgroundcolor=\color{backcolour},
    commentstyle=\color{codegreen},
    keywordstyle=\color{blue},      % 蓝色关键字
    numberstyle=\tiny\color{codegray},
    stringstyle=\color{codegreen},  % 使用与注释相同的颜色来表示字符串
    basicstyle=\ttfamily\footnotesize,
    breakatwhitespace=false,
    breaklines=true,
    captionpos=b,
    keepspaces=true,
    numbers=left,
    numbersep=5pt,
    showspaces=false,
    showstringspaces=false,
    showtabs=false,
    tabsize=2
}
\lstset{style=mystyle}
\hypersetup{
	colorlinks=true,
	linkcolor=BlueViolet,
	urlcolor=RoyalPurple,
	citecolor=Sepia
}
\newtheorem{claim}{Claim}[section]
\newtheorem{theorem}{Theorem}[section]
\newtheorem{corollary}[claim]{Corollary}
\newtheorem{lemma}[claim]{Lemma}
\newtheorem*{remark}{\textit{Remark}}
\newtheorem*{example}{\textit{Example}}
\theoremstyle{definition} % set conj style same as def style
\newtheorem{conjecture}{Conjecture}
\theoremstyle{definition}
\newtheorem{definition}{Definition}[section]



\renewenvironment{abstract}
  {\small
   \begin{center}
   \bfseries \textcolor{Fuchsia}{Abstract} \vspace{-0.5em} \vspace{0pt}
   \end{center}
   \list{}{%
     \setlength{\leftmargin}{0mm}%
     \setlength{\rightmargin}{\leftmargin}%
   }%
   \item\relax}
  {\endlist}


\setlength{\parindent}{0mm}
\setlength{\parskip}{\medskipamount}


\cleanlookdateon

\makeatletter
\newcommand{\@assignment}[0]{Assignment}
\newcommand{\assignment}[1]{\renewcommand{\@assignment}{#1}}
\newcommand{\@supervisor}[0]{}
\newcommand{\supervisor}[1]{\renewcommand{\@supervisor}{#1}}
\newcommand{\@yearofstudy}[0]{}
\newcommand{\yearofstudy}[1]{\renewcommand{\@yearofstudy}{#1}}
\makeatletter


\newtoggle{IsDissertation}
\toggletrue{IsDissertation} 
