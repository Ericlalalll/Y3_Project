\chapter{The Spectral Algorithms}\label{chapter:2}
In this chapter, we focus on the the case SNR $>1.$ Specifically, we revisit the standard spectral algorithms and non-backtracking matrix based spectral algorithm which is used to prove theorem $\ref{thm:1.1.1}.$\\
Spectral methods typically refer to the approach of partitioning the nodes into groups by using the  spectral properties of the graph matrices \cite{dallamico:tel-03454227}. The eigenvalue spectrum of various graph matrices, such as the adjacency matrix, the Laplacian, and the non-backtracking matrix, usually consists of a compact bulk of closely spaced eigenvalues, along with some outlying eigenvalues separated from the bulk. The eigenvectors corresponding to these outliers often  provide insights into the large-scale structure of the network, including its community structure \cite{userguide}. Therefore, it enables us to detect the communities in  a graph by utilising the the eigenvectors corresponding to these outlying eigenvalues.
\section{Standard Spectral Algorithm}
The standard spectral algorithms usually refer to spectral algorithms based on adjacency or Laplacian matrix. \\
It has been demonstrated in \cite{standard_spec_in_dense} that if SNR $>1$, for a given graph with built-in community structure, the eigenvector corresponding to the second largest eigenvalue of its adjacency matrix is correlated with the true community structure when the graph is sufficient dense. Therefore, the spectral algorithm based on the adjacency matrix can compute community labels by simply labelling vertices according to the sign of the second eigenvector for the case of two communities ($k=2$). This approach can be generalised to the case of more than two communities ($k>2$) by applying heuristic techniques such as the k-means algorithm.\\
However, this correlation diminishes in sparse graphs (i.e., $c$ is constant while $n\to\infty$), as detailed in \cite{the_non-backtracking}. In sparse graphs, eigenvectors corresponding to eigenvalues outside the bulk may correlate with high-degree vertices rather than the community structure. Similarly, community-related eigenvalue could be merged into the bulk, swamping the community-correlated eigenvector with uninformative eigenvectors. As a consequence, identifying the community-correlated eigenvector becomes challenging or even impossible. The same issue arises when using the Laplacian matrix.\\
One potential remedy is to remove the high-degree vertices, but this approach discards a significant amount of information and can cause the graph to break apart into disconnected components \cite{TheConjecture}. Therefore, an alternative method is required to mitigate the influence of high-degree vertices while preserving the community structure information.

\section{Spectral Algorithm based on Non-backtracking Matrix}\label{sec: Spectral Algorithm based on Non-backtracking Matrix}
The non-backtracking matrix, as defined in definition \ref{def: non_backtracking}, is introduced to address this issue. Due to its non-backtracking nature, i.e., a vertex cannot be revisited within two steps, the non-backtracking matrix is less sensitive to high-degree vertices compared to the adjacency or Laplacian matrices.\\
Let $\mu_i$ denote the \textit{i-th} largest eigenvalue of the non-backtracking matrix, $\lambda_i$ denote its corresponding eigenvector.
it was shown in \cite{the_non-backtracking} that the non-backtracking matrix has the following spectrum properties:
\begin{theorem}\label{thm: 2.2.1}
    Let $G$ be drawn from $SSBM(n, 2, p_{in}, p_{out}).$ If $(c_{in}-c_{out})/2>\sqrt{c},$ then, with high probability, the two largest eigenvalues of $\mathbf{B}$ are real and satisfy
    \begin{itemize}
     \vspace{-5mm}
        \item $\mu_1\to c$
        \vspace{-3mm}
        \item $\mu_2\to(c_{in}-c_{out})/2$
        \vspace{-3mm}
        \item $|\mu_{i_{>2}}|\leq\sqrt{c},$ i.e. the radius of the bulk of $\mathbf{B}$'s spectrum is confined in $\sqrt{c}$
    \end{itemize}
\end{theorem}
\begin{remark}
    the constraint $(c_{in}-c_{out})/2>\sqrt{c}$ is equivalent to SNR $>1.$
\end{remark}
Krzakala et al. \cite{the_non-backtracking} point out that the eigenvector $\lambda_2$ is strongly correlated with the true community structure. Moreover, according to Theorem \ref{thm: 2.2.1}, $\mu_2$, the second largest eigenvalue of the non-backtracking matrix $\mathbf{B}$, is well-separated from the bulk of $\mathbf{B}$'s spectrum when SNR > 1, regardless of whether the graph is sparse or dense. Figure \ref{fig: spectrum} illustrates an example of the spectrum of $\mathbf{B}$, showcasing this separation.
\begin{figure}
    \centering
    \includegraphics[width=1\linewidth]{Figures/spectrum_plot.pdf}
    \caption[Spectrum Distribution of the Non-Backtracking Matrix]{The spectrum of the non-backtracking matrix of a graph drawn from $SSBM$ with parameters $n=10^3,~k=2,~c_{in}=5,~c_{out}=1.$ In this scenario, SNR $=4/3$, and we can observe that $\mu_2$ is well-separated from bulk of spectrum. Moreover, $\mu_1\to3=(c_{in}+c_{out})/2=c$, $\mu_2\to2=(c_{in}-c_{out})/2$, and the radius of the bulk is $\sqrt{c}=\sqrt{3},$ These observations are in alignment with Theorem $\ref{thm: 2.2.1}.$}
    \label{fig: spectrum}
\end{figure}

Therefore, the spectral algorithm based on non-backtracking matrix validates Theorem \ref{thm:1.1.1}. The algorithmic procedure of this spectral algorithm is summarized in Algorithm \ref{algo1}.

\begin{algorithm}[H]\label{algo1}
\SetAlgoLined       % This enables vertical lines to indicate block levels
\SetKwInput{KwInput}{Input}                % Set the Input
\SetKwInput{KwOutput}{Output}              % Set the Output
\LinesNumbered
\KwInput{Graph $G=(V, E)$ drawn from $SSBM(n, 2, p_{in}, p_{out})$}
\KwOutput{Computed communities assignment $\sigma'$}
\Begin{
Convert $G$ into a directed graph;\\
$\mathbf{B}\leftarrow$ non-backtracking matrix of $G$;\\
$\xi\leftarrow$ eigenvector corresponding to the second largest eigenvalue;\\
\For{$v\in V$}{
  \If{$\sum_{\{(u,v)\in E:~\text{head is}~v\}}\xi_{(u, v)}>0$}{$\sigma_v'\leftarrow$ Community A;}
  \Else{$\sigma_v'\leftarrow$ Community B;}
}
\Return{$\sigma'$}
}
\BlankLine
\caption{Spectral Algorithm based on Non-Backtracking Matrix}
\end{algorithm}
\clearpage
Figure \ref{fig:spec_algo} presents the experimental results of Algorithm \ref{algo1}. These results are consistent with those reported in \cite{the_non-backtracking}.
\begin{figure}[H]
    \centering
    \includegraphics[width=1\linewidth]{Figures/spec_algo.jpg}
    \caption[Accuracy of the Spectral Algorithm based on Non-Backtracking Matrix]{The accuracy of spectral algorithm based on non-backtracking matrix across different SNR values. This experiment is averaged over 10 instances with $n=10^4$ and $c=3.$}
    \label{fig:spec_algo}
\end{figure}
From Figure \ref{fig:spec_algo}, we observe that when SNR > 1, the spectral algorithm based on the non-backtracking matrix noticeably outperforms random guessing, which has an expected accuracy of 50\% across all SNR values.
